\section{Hardware requirements}

\subsection{Set-up PC}
LabVIEW allows to easily program code that can be executed in parallel and HelioScan makes extensive use of this feature. Therefore, we recommend to use an up-to-date PC with at least four processor cores. For fast start-up of HelioScan and saving of acquired data, the operating system as well as HelioScan is optimally installed on a \ac{SSD}. To allow the main \ac{GUI} of HelioScan to be properly displayed, a monitor with a resolution of at least 1600x1200 pixels is required. 

\subsubsection{Example configuration}
At the time of writing, the following hardware combination provided to work well:
\begin{itemize}[noitemsep]
	\item ASUS P6T DELUXE motherboard
	\item Intel Core i7-950 \SI{3.06}{\GHz} processor (quadcore)
	\item 12 GB DDR3 \SI{1333}{\MHz} RAM
	\item NVIDIA GTX460 1GB graphics controller
	\item 1TB S-ATAII harddisk (7200 rpm)
	\item Kingston SV100S2/128G solid-state disk with 128GB
	\item DVD-+R/RW dual-layer drive
	\item PCI-Express serial port (RS-232) card
\end{itemize}

\subsection{Periphery hardware}
The type of periphery hardware needed depends on several factors. These include your particular type of microscope (e.g. which stage device it is based on), its image acquisition modality  (e.g. laser-scanning with a particular combination of scanner types), the imaging modes you want to use (e.g. 3D spiral scanning), the experiments you want to perform (you might need specific hardware for stimulators) and the available budget.